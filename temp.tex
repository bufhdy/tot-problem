1984年麥克·弗雷德曼(Michael Lawrence Fredman)和羅伯特·塔揚\footnote{Robert Endre Tarjan,1948年4月30日-,美國計算機科學家、數學家,1986年圖靈獎得主。他發明了不少圖論算法,包括求取最近公共祖先的算法、強連通分量和雙連通圖的算法。同時也參與了伸展樹、斐波那契堆的發明。}

\section{貝爾曼–福特演算法}

貝爾曼-福特演算法(Bellman–Ford algorithm)亦在有向賦權同中用以求取單元最短路徑。最開始由Alfonso Shimbel於1955年提出,在之後的1956和1958年福特\footnote{福特(Lester Randolph Ford Jr.,1927年9月23日–2017年2月26日),主攻網絡流的美國數學家。}和貝爾曼\footnote{理察·貝爾曼(Richard Bellman,1920年8月26日-1984年3月19日)美國應用數學家,美國國家科學院院士,動態規劃的創始人。他在1979年被授予電氣電子工程師協會獎,由於其在「決策過程和控制系統理論方面的貢獻,特別是動態規劃的發明和應用。」[5]他的主要工作是貝爾曼方程。},對於某些問題,它要比戴克斯特拉算法要慢,但卻更加巧妙,因爲它可以用來解決含有\emph{負邊權}的圖。